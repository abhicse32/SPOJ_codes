%
% This is the LaTeX template file for lecture notes for CS267,
% Applications of Parallel Computing.  When preparing 
% LaTeX notes for this class, please use this template.
%
% To familiarize yourself with this template, the body contains
% some examples of its use.  Look them over.  Then you can
% run LaTeX on this file.  After you have LaTeXed this file then
% you can look over the result either by printing it out with
% dvips or using xdvi.
%

\documentclass[twoside]{article}
\setlength{\oddsidemargin}{0.25 in}
\setlength{\evensidemargin}{-0.25 in}
\setlength{\topmargin}{-0.6 in}
\setlength{\textwidth}{6.5 in}%
% This is the LaTeX template file for lecture notes for CS267,
% Applications of Parallel Computing.  When preparing 
% LaTeX notes for this class, please use this template.
%
% To familiarize yourself with this template, the body contains
% some examples of its use.  Look them over.  Then you can
% run LaTeX on this file.  After you have LaTeXed this file then
% you can look over the result either by printing it out with
% dvips or using xdvi.
%

\documentclass[twoside]{article}
\setlength{\oddsidemargin}{0.25 in}
\setlength{\evensidemargin}{-0.25 in}
\setlength{\topmargin}{-0.6 in}
\setlength{\textwidth}{6.5 in}
\setlength{\textheight}{8.5 in}
\setlength{\headsep}{0.75 in}
\setlength{\parindent}{0 in}
\setlength{\parskip}{0.1 in}

%
% ADD PACKAGES here:
%

\usepackage{amsmath,amsfonts,graphicx}
\usepackage{pgfplots}
\pgfplotsset{width=10cm,compat=1.9}
\usepgfplotslibrary{external}

\tikzexternalize

%
% The following commands set up the lecnum (lecture number)
% counter and make various numbering schemes work relative
% to the lecture number.
%
\newcounter{lecnum}
\renewcommand{\thepage}{\arabic{page}}
\renewcommand{\thesection}{\arabic{section}}
\renewcommand{\theequation}{\arabic{equation}}
\renewcommand{\thefigure}{\arabic{figure}}
\renewcommand{\thetable}{\arabic{table}}

%
% The following macro is used to generate the header.
%
\newcommand{\lecture}[4]{
   \pagestyle{myheadings}
   \thispagestyle{plain}
   \newpage
   \setcounter{lecnum}{#1}
   \setcounter{page}{1}
   \noindent
   \begin{center}
   \framebox{
      \vbox{\vspace{2mm}
    \hbox to 6.28in { {\bf CS5011: Introduction to Machine Learning
    \hfill Fall 2014} }
       \vspace{4mm}
       \hbox to 6.28in { {\Large \hfill Lecture #1: #2  \hfill} }
       \vspace{2mm}
       \hbox to 6.28in { {\it Lecturer: #3 \hfill Scribes: #4} }
      \vspace{2mm}}
   }
   \end{center}
   \markboth{Lecture #1: #2}{Lecture #1: #2}

   {\bf Disclaimer}: {\it These notes have not been subjected to the
   usual scrutiny reserved for formal publications.  They may be distributed
   outside this class only with the permission of the Instructor.}
   \vspace*{4mm}
}
%
% Convention for citations is authors' initials followed by the year.
% For example, to cite a paper by Leighton and Maggs you would type
% \cite{LM89}, and to cite a paper by Strassen you would type \cite{S69}.
% (To avoid bibliography problems, for now we redefine the \cite command.)
% Also commands that create a suitable format for the reference list.
\renewcommand{\cite}[1]{[#1]}
\def\beginrefs{\begin{list}%
        {[\arabic{equation}]}{\usecounter{equation}
         \setlength{\leftmargin}{2.0truecm}\setlength{\labelsep}{0.4truecm}%
         \setlength{\labelwidth}{1.6truecm}}}
\def\endrefs{\end{list}}
\def\bibentry#1{\item[\hbox{[#1]}]}

%Use this command for a figure; it puts a figure in wherever you want it.
%usage: \fig{NUMBER}{SPACE-IN-INCHES}{CAPTION}
\newcommand{\fig}[3]{
      \vspace{#2}
      \begin{center}
      Figure #1:~#3
      \end{center}
  }
% Use these for theorems, lemmas, proofs, etc.
\newtheorem{theorem}{Theorem}
\newtheorem{lemma}[theorem]{Lemma}
\newtheorem{proposition}[theorem]{Proposition}
\newtheorem{claim}[theorem]{Claim}
\newtheorem{corollary}[theorem]{Corollary}
\newtheorem{definition}[theorem]{Definition}
\newenvironment{proof}{{\bf Proof:}}{\hfill\rule{2mm}{2mm}}

% **** IF YOU WANT TO DEFINE ADDITIONAL MACROS FOR YOURSELF, PUT THEM HERE:

\newcommand\E{\mathbb{E}}

\begin{document}
%FILL IN THE RIGHT INFO.
%\lecture{**LECTURE-NUMBER**}{**DATE**}{**LECTURER**}{**SCRIBE**}
\lecture{7}{A -Optimization}{Sarath Chandar A P}{Abhishek Yadav}
%\footnotetext{These notes are partially based on those of Nigel Mansell.}

% **** YOUR NOTES GO HERE:

% Some general latex examples and examples making use of the
% macros follow.  
%**** IN GENERAL, BE BRIEF. LONG SCRIBE NOTES, NO MATTER HOW WELL WRITTEN,
%**** ARE NEVER READ BY ANYBODY.
\begin{definition}\label{def:min1}
: A point $x^*$ is a  \underline{ $global$} minimizer of f if for any other point x, \( f(x^*) \leq f(x)\).
\end{definition}
\begin{definition}\label{def:min2}
: A point $x^*$ is a  \underline{$local$} minimizer of f if there is a neighborhood U of $x^*$, such that for any point \( x \in U\), \(f(x^*) \leq f(x)\).
\end{definition}
\begin{definition}\label{def:min3}
: A point $x^*$ is a  \underline{strict local} minimizer of f if there is a neighborhood U of $x^*$, such that for any point \( x \in U\), \(f(x^*) < f(x)\).
\end{definition}

\begin{definition}\label{def:min4}
: A point $x^*$ is an  \underline{isolated local} minimizer of f if there is a neighborhood U of $x^*$, such that there is no other local minimizers in this neighborhood.
\end{definition}

\section{Unconstrained Optimization} % Don't be this informal in your notes!
 An unconstrained optimization problem, has the form:\\
 
 \begin{equation} \label{eq:func1}
 {\it minimize \hspace*{.05in} f(x)} \\
\end{equation}

\begin{text}
        where {\it f } : ${\bf R^n} \rightarrow {\bf R}$ is convex and twice continuously differentiable (which implies that ${\bf dom }$ ${\it f}$ is open).
We will assume that problem is solvable, {\it i.e.}, there exists an optimal unique point $x^*$. We denote the optimal value, $inf_x$ \textit{$f(x)$ = $f(x^*)$}, as $p^*$.\\
        \hspace*{5mm} Since ${\it f}$ is differentiable and convex, a necessary and sufficient condition for a point $x^*$ to be optimal is
\begin{equation}\label{eq:func2}
    \nabla f(x^*) = 0 
\end{equation}
Thus solving  the unconstrained optimization problem \eqref{eq:func1} is the same as finding a solution of \eqref{eq:func2} which is a set of n equations in the n variables \(x_1,x_2,\ldots ,x_n\). 
\end{text}
\subsection{Example}
\textbf{ Quadratic minimization and least squares}\\
\begin{text}
The general convex quadratic minimization problem has the form
\begin{equation}\label{eq:three}
  minimize \hspace{0.05in}(1/2)x^{T}Px+q^{T}+r
\end{equation}
where \(P \in {\bf S_+^n}\), \(q \in {\bf R^n}\), and \(r \in {\bf R}\). This problem can be solved via the optimality conditions \(Px^* + q = 0\), which is a set of linear equations. When \(P \succ 0\), there is a unique solution \(x^* = -P^-1q\). In the more general case when P is not positive definite, any solution of \(Px^* = -q\) is optimal for \eqref{eq:three}; if \(Px^* = -q\) does not have a solution, then the problem \eqref{eq:three} is unbounded below.\\
\hspace*{5mm} One special of the quadratic minimization problem that arises very frequently is the least-square problem \[ minimize \hspace*{0.1in} \| Ax - b \|_2^2 = x^T(A^TA)x - 2(A^Tb)^Tx + b^Tb . \] The optimality conditions \[A^TAx^* = A^Tb\] are called the {\it normal equations} of the least squares problem.
\end{text}
\section{Constrained Optimization}
\subsection{Optimization Problem}
An optimization problem is described as
\[minimize \hspace*{0.05in}f_0(x)\]
\[subject \hspace*{0.05in} to\]
\[f_i(x) \leq 0, i = 1,2,3,\ldots,m\]
\[h_i(x) = 0, i = 1,2,3,\ldots,p\]
 \begin{text}
 where,\\
 \(x \in {\bf R^n}\) is known as the optimization variable\\
 $f_0$ is the objective function\\
 \(f_i(x) \leq 0\) form the inequality constraints\\
 \(f_i : {\bf R^n} \rightarrow {\bf R}\) are the inequality constraint functions\\
 \(h_i(x) = 0\) are the equality constraint functions\\
 \(h_i : {\bf R^n} \rightarrow {\bf R}\) are the equality constraint functions\\
 \[\mathbb{D} = \left(\cap_{i=0}^m {\bf dom }\hspace{0.5mm} f_i\right) \cap \left(\cap_{i=0}^P {\bf dom }\hspace{0.5mm} h_i\right)\]
 represents the set of points for which the objective and all constraint functions are defined and is known as the domain of the optimization problem. A point \(x \in D\) is feasible if it satisfies the equality and inequality
constraints. A problem is feasible if there exists atleast one feasible point, and infeasible otherwise. The feasible set of an optimization problems is defined as the set of all feasible points.
\end{text}
\subsection{Optimal value of a function}
\begin{text}
the optimal value \(p^* = inf \{f_0(x)|f_i(x) \leq 0,i = 1\ldots$m$, h_i(x) = 0, i = 1\ldots$p$, f_0(x) = p^* \}\)\\ \\
Take all the feasible points and find the infimum. This forms the optimal value. It should be noted that we do not use the minimum in the definition of the optimal value because minimum value might not exist in the set.
 \end{text}
\subsection{Optimal Point}
 $x^*$ is an optimal point if 
 \begin{enumerate}
 \item $x^*$ is feasible
 \item \(f_0(x) = p^*\)
 \end{enumerate}
 The set of all optimal points forms the optimal set and is denoted as,\\
 \[  {\bf \it X_{opt}} = \{x|f_i(x) \leq 0,i = 1\ldots{m}, h_i(x) = 0, i = 1\ldots{p} \} \]
 If there exists an optimal point for the problem (equation 6), we say the optimal value is attained or achieved, and the problem is solvable.
\subsection{Locally Optimal Points}
A point is locally optimal if it is optimal over its neighbours. Mathematically, a {\bf feasible point} x is locally optimal when x solves the optimization problem(with variable z and radius R),
\[minimize \hspace*{0.05in}f_0(z)\]
\[subject \hspace*{0.05in} to\]
\[f_i(z) \leq 0, i = 1,2,3,\ldots,m\]
\[h_i(z) = 0, i = 1,2,3,\ldots,p\]
\[\|z-x\|_2 \leq R,\]
The last constraint above means that we need not consider the non-neighbourhood points.
\section{Expressing problems in standard form}
\begin{text}
To find a solution for the optimization problem, first, we formulate the given problem into an optimization problem. Then it can be fed into a solver to obtain the solution.Generally solvers do not accept the problems of the following forms:
\[minimize\hspace*{0.05in}f_0(x)\]
\[subject \hspace*{0.05in}to\]
\[l_i \leq x_i \leq u_i, i = 1\ldots{n} \]
to solve such problems, they are converted to the following {\bf \it standard form}:\\
\[minimize\hspace*{0.05in}f_0(x)\]
\[subject \hspace*{0.05in}to\]
\[l_i - x_i \leq 0, i =1\ldots{n}\]
\[x_i - u_i \leq 0, i =1\ldots{n}\]
\end{text}

\section{Slack Variables}
The goal of utilizing slack variables is to change the inequalities to equalities. We do this by adding some unknown amount to the left hand side of each inequality. So a constraint \(f_i(x) \leq 0\) will be satisfied iff \(\exists s_i \geq \) such that \(f_i(x) + s_i = 0 \). This leads to the transformation of the problem into the following form:\\
\[minimize\hspace*{0.05in}f_0(x)\]
\[subject \hspace*{0.05in}to\]
\[s_i \geq 0, i = 1\ldots{n}\]
\[f_i(x) + s_i = 0, i = 1\ldots{m}\]
\[h_i(x) = 0, i = 1\ldots{p}\]
{\bf In this form problem contains n+m variables then how is it better?}: And the answer is, in this form we have only equality and non-negative constraints. Efficient algorithms are available to solve this kind of problems(which do not involve equality constraints). Hence it is better. 

\section{Epigraph Problem Form}
The {\bf epigraph} or {\bf supergraph} of a function \(f : \mathbb{R^n} \rightarrow \mathbb{R}\) is the set of points lying on or above its graph.
\[{\bf epi}\hspace{0.5mm} f = \{(x,\mu) : x \in \mathbb{R^n}, \mu \in \mathbb{R}, \mu \geq f(x)\}\subseteq R^{n+1}\]
The {\bf strict epigraph} is the epigraph with the graph itself removed.
\[{\bf epi_S}\hspace{0.5mm}f = \{(x,\mu) : x \in \mathbb{R^n}, \mu \in \mathbb{R}, \mu > f(x)\}\subseteq R^{n+1}\]
The epigraph form of the problem is given by,
\[minimize\hspace*{0.05in}f_0(x)\]
\[subject \hspace*{0.05in}to\]
\[f_0(x) - t \leq 0\]
\[f_i(x) \leq 0, i = 1\ldots{m}\]
\[h_i(x) = 0, i = 1\ldots{p}\]
with variables \(x \in R^n\) and \(t \in \mathbb{R}\). It is easy to see that it is equivalent to the original problem; a point (x,t) is optimal for the epigraph form iff x is optimal for the original problem and \(t = f_0(x)\). Note that the objective function for the epigraph form problem is a {\it linear} function of ({\bf x,t}) \\
\\
\hspace{2in}\begin{tikzpicture}
\begin{axis}
    \addplot+[mark=none,fill=blue,draw=black] {min(x^2 -(\infty),0)} \closedcycle;
\addlegendentry{$x^2$}
\end{axis}
\end{tikzpicture}\\
\begin{text}
   shaded region shows the epigraph of the function \bf \(f(x) = x^2 , x \in \mathbb{R} \)
\end{text}
\section{Feasibility Problem}
Often times a constrained optimization problem requires an initial point, which is feasible in the sense that it satisfies all the constraints. The problem of finding such a point is referred to as the {\it feasibility problem}
 \[find \hspace*{0.05in} x\]
 \[subject \hspace*{0.05in} to\]
 \[f_i(x) \leq 0, i = 1,2,3,\ldots,m\]
\[h_i(x) = 0, i = 1,2,3,\ldots,p\]

The feasibility problem can be considered a special case of the general problem with \(f_0(x) = 0\), namely\\
\[minimize\hspace*{0.05in} 0\]
\[subject \hspace*{0.05in} to\]
\[f_i(x) \leq 0, i = 1\ldots{m}\]
\[h_i(x) = 0, i = 1\ldots{p}\]

\begin{itemize}
\item We have \(p^* = 0\) if the constraints are feasible; any feasible x is optimal in this case.
\item We have \(p^* = \infty \) if the constraints are infeasible.
\end{itemize}

\section*{References}
\beginrefs
\bibentry{1}{\sc Stephen ~Boyd } and {\sc Lieven ~Vandenberghe}, 
``Convex Optimization,''{\it Cambridge University Press },
\bibentry{2} Class Notes
\endrefs

% **** THIS ENDS THE EXAMPLES. DON'T DELETE THE FOLLOWING LINE:

\end{document}





\setlength{\textheight}{8.5 in}
\setlength{\headsep}{0.75 in}
\setlength{\parindent}{0 in}
\setlength{\parskip}{0.1 in}

%
% ADD PACKAGES here:
%

\usepackage{amsmath,amsfonts,graphicx}
\usepackage{pgfplots}
\pgfplotsset{width=10cm,compat=1.9}
\usepgfplotslibrary{external}

\tikzexternalize

%
% The following commands set up the lecnum (lecture number)
% counter and make various numbering schemes work relative
% to the lecture number.
%
\newcounter{lecnum}
\renewcommand{\thepage}{\arabic{page}}
\renewcommand{\thesection}{\arabic{section}}
\renewcommand{\theequation}{\arabic{equation}}
\renewcommand{\thefigure}{\arabic{figure}}
\renewcommand{\thetable}{\arabic{table}}

%
% The following macro is used to generate the header.
%
\newcommand{\lecture}[4]{
   \pagestyle{myheadings}
   \thispagestyle{plain}
   \newpage
   \setcounter{lecnum}{#1}
   \setcounter{page}{1}
   \noindent
   \begin{center}
   \framebox{
      \vbox{\vspace{2mm}
    \hbox to 6.28in { {\bf CS5011: Introduction to Machine Learning
		\hfill Fall 2014} }
       \vspace{4mm}
       \hbox to 6.28in { {\Large \hfill Lecture #1: #2  \hfill} }
       \vspace{2mm}
       \hbox to 6.28in { {\it Lecturer: #3 \hfill Scribes: #4} }
      \vspace{2mm}}
   }
   \end{center}
   \markboth{Lecture #1: #2}{Lecture #1: #2}

   {\bf Disclaimer}: {\it These notes have not been subjected to the
   usual scrutiny reserved for formal publications.  They may be distributed
   outside this class only with the permission of the Instructor.}
   \vspace*{4mm}
}
%
% Convention for citations is authors' initials followed by the year.
% For example, to cite a paper by Leighton and Maggs you would type
% \cite{LM89}, and to cite a paper by Strassen you would type \cite{S69}.
% (To avoid bibliography problems, for now we redefine the \cite command.)
% Also commands that create a suitable format for the reference list.
\renewcommand{\cite}[1]{[#1]}
\def\beginrefs{\begin{list}%
        {[\arabic{equation}]}{\usecounter{equation}
         \setlength{\leftmargin}{2.0truecm}\setlength{\labelsep}{0.4truecm}%
         \setlength{\labelwidth}{1.6truecm}}}
\def\endrefs{\end{list}}
\def\bibentry#1{\item[\hbox{[#1]}]}

%Use this command for a figure; it puts a figure in wherever you want it.
%usage: \fig{NUMBER}{SPACE-IN-INCHES}{CAPTION}
\newcommand{\fig}[3]{
			\vspace{#2}
			\begin{center}
			Figure #1:~#3
			\end{center}
	}
% Use these for theorems, lemmas, proofs, etc.
\newtheorem{theorem}{Theorem}
\newtheorem{lemma}[theorem]{Lemma}
\newtheorem{proposition}[theorem]{Proposition}
\newtheorem{claim}[theorem]{Claim}
\newtheorem{corollary}[theorem]{Corollary}
\newtheorem{definition}[theorem]{Definition}
\newenvironment{proof}{{\bf Proof:}}{\hfill\rule{2mm}{2mm}}

% **** IF YOU WANT TO DEFINE ADDITIONAL MACROS FOR YOURSELF, PUT THEM HERE:

\newcommand\E{\mathbb{E}}

\begin{document}
%FILL IN THE RIGHT INFO.
%\lecture{**LECTURE-NUMBER**}{**DATE**}{**LECTURER**}{**SCRIBE**}
\lecture{7}{A -Optimization}{Sarath Chandar A P}{Abhishek Yadav}
%\footnotetext{These notes are partially based on those of Nigel Mansell.}

% **** YOUR NOTES GO HERE:

% Some general latex examples and examples making use of the
% macros follow.  
%**** IN GENERAL, BE BRIEF. LONG SCRIBE NOTES, NO MATTER HOW WELL WRITTEN,
%**** ARE NEVER READ BY ANYBODY.
\begin{definition}\label{def:min1}
: A point $x^*$ is a  \underline{ $global$} minimizer of f if for any other point x, \( f(x^*) \leq f(x)\).
\end{definition}
\begin{definition}\label{def:min2}
: A point $x^*$ is a  \underline{$local$} minimizer of f if there is a neighborhood U of $x^*$, such that for any point \( x \in U\), \(f(x^*) \leq f(x)\).
\end{definition}
\begin{definition}\label{def:min3}
: A point $x^*$ is a  \underline{strict local} minimizer of f if there is a neighborhood U of $x^*$, such that for any point \( x \in U\), \(f(x^*) < f(x)\).
\end{definition}

\begin{definition}\label{def:min4}
: A point $x^*$ is an  \underline{isolated local} minimizer of f if there is a neighborhood U of $x^*$, such that there is no other local minimizers in this neighborhood.
\end{definition}

\section{Unconstrained Optimization} % Don't be this informal in your notes!
 An unconstrained optimization problem, has the form:\\
 
 \begin{equation} \label{eq:func1}
 {\it minimize \hspace*{.05in} f(x)} \\
\end{equation}

\begin{text}
        where {\it f } : ${\bf R^n} \rightarrow {\bf R}$ is convex and twice continuously differentiable (which implies that ${\bf dom }$ ${\it f}$ is open).
We will assume that problem is solvable, {\it i.e.}, there exists an optimal unique point $x^*$. We denote the optimal value, $inf_x$ \textit{$f(x)$ = $f(x^*)$}, as $p^*$.\\
        \hspace*{5mm} Since ${\it f}$ is differentiable and convex, a necessary and sufficient condition for a point $x^*$ to be optimal is
\begin{equation}\label{eq:func2}
    \nabla f(x^*) = 0 
\end{equation}
Thus solving  the unconstrained optimization problem \eqref{eq:func1} is the same as finding a solution of \eqref{eq:func2} which is a set of n equations in the n variables \(x_1,x_2,\ldots ,x_n\). 
\end{text}
\subsection{Example}
\textbf{ Quadratic minimization and least squares}\\
\begin{text}
The general convex quadratic minimization problem has the form
\begin{equation}\label{eq:three}
  minimize \hspace{0.05in}(1/2)x^{T}Px+q^{T}+r
\end{equation}
where \(P \in {\bf S_+^n}\), \(q \in {\bf R^n}\), and \(r \in {\bf R}\). This problem can be solved via the optimality conditions \(Px^* + q = 0\), which is a set of linear equations. When \(P \succ 0\), there is a unique solution \(x^* = -P^-1q\). In the more general case when P is not positive definite, any solution of \(Px^* = -q\) is optimal for \eqref{eq:three}; if \(Px^* = -q\) does not have a solution, then the problem \eqref{eq:three} is unbounded below.\\
\hspace*{5mm} One special of the quadratic minimization problem that arises very frequently is the least-square problem \[ minimize \hspace*{0.1in} \| Ax - b \|_2^2 = x^T(A^TA)x - 2(A^Tb)^Tx + b^Tb . \] The optimality conditions \[A^TAx^* = A^Tb\] are called the {\it normal equations} of the least squares problem.
\end{text}
\section{Constrained Optimization}
\subsection{Optimization Problem}
An optimization problem is described as
\[minimize \hspace*{0.05in}f_0(x)\]
\[subject \hspace*{0.05in} to\]
\[f_i(x) \leq 0, i = 1,2,3,\ldots,m\]
\[h_i(x) = 0, i = 1,2,3,\ldots,p\]
 \begin{text}
 where,\\
 \(x \in {\bf R^n}\) is known as the optimization variable\\
 $f_0$ is the objective function\\
 \(f_i(x) \leq 0\) form the inequality constraints\\
 \(f_i : {\bf R^n} \rightarrow {\bf R}\) are the inequality constraint functions\\
 \(h_i(x) = 0\) are the equality constraint functions\\
 \(h_i : {\bf R^n} \rightarrow {\bf R}\) are the equality constraint functions\\
 \[\mathbb{D} = \left(\cap_{i=0}^m {\bf dom }\hspace{0.5mm} f_i\right) \cap \left(\cap_{i=0}^P {\bf dom }\hspace{0.5mm} h_i\right)\]
 represents the set of points for which the objective and all constraint functions are defined and is known as the domain of the optimization problem. A point \(x \in D\) is feasible if it satisfies the equality and inequality
constraints. A problem is feasible if there exists atleast one feasible point, and infeasible otherwise. The feasible set of an optimization problems is defined as the set of all feasible points.
\end{text}
\subsection{Optimal value of a function}
\begin{text}
the optimal value \(p^* = inf \{f_0(x)|f_i(x) \leq 0,i = 1\ldots$m$, h_i(x) = 0, i = 1\ldots$p$, f_0(x) = p^* \}\)\\ \\
Take all the feasible points and find the infimum. This forms the optimal value. It should be noted that we do not use the minimum in the definition of the optimal value because minimum value might not exist in the set.
 \end{text}
\subsection{Optimal Point}
 $x^*$ is an optimal point if 
 \begin{enumerate}
 \item $x^*$ is feasible
 \item \(f_0(x) = p^*\)
 \end{enumerate}
 The set of all optimal points forms the optimal set and is denoted as,\\
 \[  {\bf \it X_{opt}} = \{x|f_i(x) \leq 0,i = 1\ldots{m}, h_i(x) = 0, i = 1\ldots{p} \} \]
 If there exists an optimal point for the problem (equation 6), we say the optimal value is attained or achieved, and the problem is solvable.
\subsection{Locally Optimal Points}
A point is locally optimal if it is optimal over its neighbours. Mathematically, a {\bf feasible point} x is locally optimal when x solves the optimization problem(with variable z and radius R),
\[minimize \hspace*{0.05in}f_0(z)\]
\[subject \hspace*{0.05in} to\]
\[f_i(z) \leq 0, i = 1,2,3,\ldots,m\]
\[h_i(z) = 0, i = 1,2,3,\ldots,p\]
\[\|z-x\|_2 \leq R,\]
The last constraint above means that we need not consider the non-neighbourhood points.
\section{Expressing problems in standard form}
\begin{text}
To find a solution for the optimization problem, first, we formulate the given problem into an optimization problem. Then it can be fed into a solver to obtain the solution.Generally solvers do not accept the problems of the following forms:
\[minimize\hspace*{0.05in}f_0(x)\]
\[subject \hspace*{0.05in}to\]
\[l_i \leq x_i \leq u_i, i = 1\ldots{n} \]
to solve such problems, they are converted to the following {\bf \it standard form}:\\
\[minimize\hspace*{0.05in}f_0(x)\]
\[subject \hspace*{0.05in}to\]
\[l_i - x_i \leq 0, i =1\ldots{n}\]
\[x_i - u_i \leq 0, i =1\ldots{n}\]
\end{text}

\section{Slack Variables}
The goal of utilizing slack variables is to change the inequalities to equalities. We do this by adding some unknown amount to the left hand side of each inequality. So a constraint \(f_i(x) \leq 0\) will be satisfied iff \(\exists s_i \geq \) such that \(f_i(x) + s_i = 0 \). This leads to the transformation of the problem into the following form:\\
\[minimize\hspace*{0.05in}f_0(x)\]
\[subject \hspace*{0.05in}to\]
\[s_i \geq 0, i = 1\ldots{n}\]
\[f_i(x) + s_i = 0, i = 1\ldots{m}\]
\[h_i(x) = 0, i = 1\ldots{p}\]
{\bf In this form problem contains n+m variables then how is it better?}: And the answer is, in this form we have only equality and non-negative constraints. Efficient algorithms are available to solve this kind of problems(which do not involve equality constraints). Hence it is better. 

\section{Epigraph Problem Form}
The {\bf epigraph} or {\bf supergraph} of a function \(f : \mathbb{R^n} \rightarrow \mathbb{R}\) is the set of points lying on or above its graph.
\[{\bf epi}\hspace{0.5mm} f = \{(x,\mu) : x \in \mathbb{R^n}, \mu \in \mathbb{R}, \mu \geq f(x)\}\subseteq R^{n+1}\]
The {\bf strict epigraph} is the epigraph with the graph itself removed.
\[{\bf epi_S}\hspace{0.5mm}f = \{(x,\mu) : x \in \mathbb{R^n}, \mu \in \mathbb{R}, \mu > f(x)\}\subseteq R^{n+1}\]
The epigraph form of the problem is given by,
\[minimize\hspace*{0.05in}f_0(x)\]
\[subject \hspace*{0.05in}to\]
\[f_0(x) - t \leq 0\]
\[f_i(x) \leq 0, i = 1\ldots{m}\]
\[h_i(x) = 0, i = 1\ldots{p}\]
with variables \(x \in R^n\) and \(t \in \mathbb{R}\). It is easy to see that it is equivalent to the original problem; a point (x,t) is optimal for the epigraph form iff x is optimal for the original problem and \(t = f_0(x)\). Note that the objective function for the epigraph form problem is a {\it linear} function of ({\bf x,t}) \\
\\
\hspace{2in}\begin{tikzpicture}
\begin{axis}
    \addplot+[mark=none,fill=blue,draw=black] {min(x^2 -(\infty),0)} \closedcycle;
\addlegendentry{$x^2$}
\end{axis}
\end{tikzpicture}\\
\begin{text}
   shaded region shows the epigraph of the function \bf \(f(x) = x^2 , x \in \mathbb{R} \)
\end{text}
\section{Feasibility Problem}
Often times a constrained optimization problem requires an initial point, which is feasible in the sense that it satisfies all the constraints. The problem of finding such a point is referred to as the {\it feasibility problem}
 \[find \hspace*{0.05in} x\]
 \[subject \hspace*{0.05in} to\]
 \[f_i(x) \leq 0, i = 1,2,3,\ldots,m\]
\[h_i(x) = 0, i = 1,2,3,\ldots,p\]

The feasibility problem can be considered a special case of the general problem with \(f_0(x) = 0\), namely\\
\[minimize\hspace*{0.05in} 0\]
\[subject \hspace*{0.05in} to\]
\[f_i(x) \leq 0, i = 1\ldots{m}\]
\[h_i(x) = 0, i = 1\ldots{p}\]

\begin{itemize}
\item We have \(p^* = 0\) if the constraints are feasible; any feasible x is optimal in this case.
\item We have \(p^* = \infty \) if the constraints are infeasible.
\end{itemize}

\section*{References}
\beginrefs
\bibentry{1}{\sc Stephen ~Boyd } and {\sc Lieven ~Vandenberghe}, 
``Convex Optimization,''{\it Cambridge University Press },
\bibentry{2} Class Notes
\endrefs

% **** THIS ENDS THE EXAMPLES. DON'T DELETE THE FOLLOWING LINE:

\end{document}




